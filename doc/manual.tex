\documentclass[12pt,a4paper]{article}
\usepackage[utf8]{inputenc}
\usepackage[pdftex]{hyperref}
\usepackage{makeidx,fancyhdr,graphicx,times,multicol,comment}
\usepackage[T1]{fontenc}
\usepackage{listingsutf8}
\usepackage[a4paper,total={7in,8in}]{geometry}



\title{Manual for VirusFarm}
\author{David Brown\\Tatyana Dobreva\\Pétur Helgi Einarsson}

\pagestyle{fancy}
\lhead{}
\chead{}
\rhead{\thepage}
\cfoot{}

\begin{document}
\sloppy
\maketitle

\begin{abstract}
The VirusFarm software is the product of a SURF project at the CLOVER center 
under the mentorship of Dr. Viviana Gradinaru and Dr. Benjamin E. Deverman 
during the summer 2016. This software was built to speed up the process of
 discovering new viral vectors with the CREATE selection method.
\end{abstract}

\tableofcontents

\newpage

\section{The Pipeline}

\newpage
\section{Aligning the Data}

\subsection{Requirements}
In order to run alignment, the following is required:

\begin{itemize}
    \item At least one sample is defined and non-empty of files in FASTQ format.
    \item Template sequence for each sample to be aligned
    \item One or more alignment is enabled
\end{itemize}

Template IDs are running integers starting at 1. To view templates that are 
already in the database, click 'View Existing Templates'. When clicked, a 
window pops up on the screen displaying all templates in the database and their 
IDs. 

\subsection{Alignment methods}

\subsubsection{Perfect Match Alignment}
The perfect match alignment matches a sequence read to the given template 
sequence nucleotide by nucleotide. The perfect match alignment method allows 
ambiguous characters in the template sequence.

\vspace{0.5cm}
\begin{table}[h!]
    \centering
    \begin{tabular}{|l|l|p{8cm}|}
        \hline
        \textbf{Parameter} & \textbf{Type} & \textbf{Meaning} \\
        \hline
        Mismatch Quality Threshold & Integer & The minimum read quality for a 
        nucleotide read in a non-variant position\\
        \hline
        Variant Quality Threshold & Integer & The minimum read quality for a 
        nucleotide read in a variant position.\\
        \hline
    \end{tabular}
    \caption{Parameters for Perfect Match alignment}
    \label{perfect_match_parameters}
\end{table}

\subsubsection{BowTie2}
This method is still in progress. There are several approaches that can be used
 with this method.

\vspace{0.5cm}
\noindent
\underline{\textbf{Eliminating variants from template}}:

This method removes the variants from the template sequence, making it shorter.
 The sample is the aligned to the new template. The BowTie algorithm should 
 then suggest inserts in the positions where the variants were originally with
  a possibility of other insertions/deletions. This is not guaranteed to work 
  but often gives better results than perfect match. The new inserts are then 
  examined and are checked if they match with the form of the variants 
  (NNK, NNN etc).

\vspace{0.5cm}
\noindent
\underline{\textbf{Using wildcards}}:

This approach uses Bowtie's ability to align with ambiguous characters. Since 
BowTie interprets every character other than A, C, G and T to be Ns, this 
method aligns to the template sequence with a small alternation. If there is a 
K in a variant, it is substituted by a T. Bowtie will allow this but will 
possibly indicate a mismatch at that location. If the mismatch shows a G in the
 sequence read, the sequence matches. Similar approach is used for other IUPAC 
 characters.

\begin{table}[h!]
    \centering
    \begin{tabular}{|l|l|l|p{8cm}|}
    \hline
        \textbf{Parameter} &\textbf{Type}&\textbf{Meaning}  \\
        \hline
         Local Alignment & Check box & Determines whether the alignment should 
         be \textit{local} or \textit{global}\\
         \hline
         Approach & Radio button & Selects for which approach should be used.\\
         \hline
    \end{tabular}
    \caption{Parameters for BowTie2 alignment}
    \label{bowtie_parameters}
\end{table}

\subsection{Realigning}

\subsection{Output}
When an alignment method has finished aligning to a sample, the alignment 
statistics are written to the database on disc. Under the 'Alignment Statistics'
 tab, the user can watch the statistics as they are generated. A progress 
 message is displayed above the 'Start Alignment' button.

\subsection{IUPAC Ambiguity Codes}

IUPAC ambiguity codes are used to express a character that matches more than 
one nucleotide. The table can be seen below.

\begin{table}[h!]
    \centering
    \begin{tabular}{|c|c|c|}
    \hline
    IUPAC Code & Meaning & Complement\\
\hline
A & A & T\\
\hline
C & C & G\\
\hline
G & G & C\\
\hline
T & T & A\\
\hline
M & A or C & K\\
\hline
R & A or G & Y\\
\hline
W & A or T & W\\
\hline
S & C or G & S\\
\hline
Y & C or T & R\\
\hline
K & G or T & M\\
\hline
V & A or C or G & B\\
\hline
H & A or C or T & D\\
\hline
D & A or G or T & H\\
\hline
B & C or G or T & V\\
\hline
N & A or C or G or T& N\\
\hline
    \end{tabular}
    \caption{Source: 
    \url{http://droog.gs.washington.edu/parc/images/iupac.html}}
    \label{tab:my_label}
\end{table}

\section{Analysis}

\subsection{Some analysis}

\end{document}
